\documentclass{article}
\usepackage[utf8]{inputenc}
\usepackage{geometry}     
\usepackage{latexsym}
\geometry{left=2.5cm,right=2.5cm,top=2.5cm,bottom=2.5cm}
\usepackage{fancyhdr}
\pagestyle{fancy}
\usepackage{amssymb}
\title{CS6003-INET Foundation of Computer Science Problem Set3}
\author{Robert Wen (robert.wen@nyu.edu), NetID: qw476, N12246277}
\date{September 2015}

\begin{document}
\begin{large}

\maketitle


\textbf{1. Suppose you wish to use the Principle of Mathematical Induction to prove that
1$\cdot$1!+2$\cdot$2!+3$\cdot$3! + $\cdots$ +n$\cdot$n!=(n+1)! - 1 for all n $\ge$ 1.}

Write P(1) (b) Write P(5) (c) Write P(k) (d) Write P(k+1) (e) Use the Principle of Mathematical Induction to prove that P(n) is true for all n

\textbf{Answer:}\\

(a) P(1) = 1$\cdot$1! = 1

(b) P(5) = 1$\cdot$1! + 2$\cdot$2! + 3$\cdot$3! + 4$\cdot$4! + 5$\cdot$5!
         = 1$\cdot$1  + 2$\cdot$2  + 3$\cdot$6  + 4$\cdot$24 + 5$\cdot$120
         = 719

(c) P(k) = 1$\cdot$1! + 2$\cdot$2! + 3$\cdot$3! + ... + k$\cdot$k!

(d) P(k+1) = 1$\cdot$1! + 2$\cdot$2! + 3$\cdot$3! + ... + k$\cdot$k! + (k+1)$\cdot$(k+1)!

(e) For n = 1, P(1) = 1$\cdot$1! = 1 = (1+1)! - 1 = 2 - 1 = 1

\indent    Suppose when n = k and k $\ge$ 1, P(k) = 1$\cdot$1! + $\cdots$ + k$\cdot$k! = (k+1)! - 1

\indent    We know P(k+1) = P(k) + (k+1)$\cdot$(k+1)! 

\indent \indent           = (k+1)! - 1 + (k+1)$\cdot$(k+1)!

\indent \indent           = (k+1+1)(k+1)! - 1

\indent \indent           = (k+2)! - 1

\indent \indent           = ((k+1)+1)! - 1

\indent    So for all n where n $\ge$ 1, 1$\cdot$1!+2$\cdot$2!+3$\cdot$3! + $\cdots$ +n$\cdot$n!=(n+1)! - 1\\

\textbf{2. Use the Principle of Mathematical Induction to prove that $n^{3}>n^{2}+3$ for all $n\ge 2$.} 

\textbf{Answer:}\\

For n = 2, $n^3$ = 8 > $n^{2} + 3$ = 7

Suppose n = k, we have $k^{3} > k^{2} + 3$

When n = k+1, we have $(k+1)^3$ = $k^3$ + 3$k^2$ + 3k + 1

So we have  $k^{3}$ + 3$k^2$ + 3k + 1 $>$ $k^{2}$ + 3 + 3$k^2$ + 3k + 1

\indent\indent \indent        $(k+1)^3 > 4k^2$ + 3k + 4

As 4$k^2$ + 3k + 4 = $k^2 + 2k + 1 + 3 + (3k^2 + k)$

\indent\indent \indent                  = $(k+1)^2 + 3 + (3k^2 + k)$
                   
As k $\ge$ 2, we have $(3k^2 + k) > 0 $

So $(k+1)^3 > 4k^2 + 3k + 4 > (k+1)^2 + 3$

So we have $(k+1)^3 > (k+1)^2 + 3$

Therefore for all $n\ge 2$, $n^{3}>n^{2}+3$.\\

\textbf{3. Use the Principle of Mathematical Induction to prove that 1+3+9+27+ $\cdots$ + $3^n$ = $\frac{3^(n+1)-1}{2}$ for all n $\ge$ 0}

\textbf{Answer:}\\


For n = 0, left side = 1, right side = $\frac{3^(0+1)-1}{2}$ = $\frac{3-1}{2}$ = 1. The equality meets

For n = 1, left side = 1 + 3 = 4, right side = $\frac{3^(1+1)-1}{2}$ = 4. The equality meets.

For n = k, suppose 1+3+ $\cdots$ + $3^k$ = $\frac{3^(k+1)-1}{2}$

When n = k+1, (1+3+ $\cdots$ + $3^k$) + $3^(k+1)$ = $\frac{3^(k+1)-1}{2} + 3^(k+1)$

 \indent\indent           = $\frac{3^(k+1)-1+2*3^(k+1)}{2}$
            
  \indent\indent          = $\frac{3*3^(k+1)-1}{2}$
            
   \indent\indent         = $\frac{3^((k+1)+1)-1}{2}$
            
So for all n where n $\ge$ 0, 1+3+9+27+ $\cdots$ + $3^n$ =$\frac{3^(n+1)-1}{2}.$\\

\textbf{4. Use the Principle of Mathematical Induction to prove that any integer amount of postage from 18 cents on up can be made from an infinite supply of 4-cent and 7-cent stamps.}

\textbf{Answer:}\\


Let us remodel the problem as for any n where n $\ge$ 18, there exists a, b $\in$ N, n = 4a + 7b

For n = 18, n = 18 = 4*1 + 7*2, where a = 1, b = 2

For n = 19, n = 19 = 4*3 + 7*1, where a = 3, b = 1

For n = 20, n = 20 = 4*5, where a = 5, b = 0

For n = 21, n = 21 = 7*3, where a = 0, b = 3

Suppose for n = k, we have k = 4a + 7b

Then k + 1 = 4a + 7b + 1

When b $\ge$ 1, k + 1 = 4a + 7(b-1) + 8 = 4(a+2) + 7(b-1)

When b = 0, to let 4a + 7b $\ge$ 18, a must be at least 5 because when a = 4 and b = 0, n = 16 which is less than 18.

So when b = 0, k + 1 = 4a + 1 = 4(a - 5) + 21 = 4(a-5) + 7*3 where a $\ge$ 5

So when n = k + 1, we can find a, b to let n = 4a + 7b

Therefore for any n $\ge$ 18, there exists a, b $\in$ N, n = 4a + 7b.\\


\textbf{5. Prove that $\sum\limits_{j=n}^{2n-1}(2j+1)=3n^{2}$ for all positive integers \emph{n}.}

\textbf{Answer:}\\


For n = 1, left side = $\sum\limits_{j=n}^{2n-1}(2j+1) = 2 * 1 + 1 = 3$

           right side = $3*1^2 = 3$
           
For n = 2, left side = $\sum\limits_{j=n}^{2n-1}(2j+1) = 4 + 1 + 6 + 1 = 12$

           right side = $3*2^2 = 12$
           
Suppose for n = k, $\sum\limits_{j=k}^{2k-1}(2j+1)=3k^{2}$

for n = k+1, we have $\sum\limits_{j=k+1}^{2k+1}(2j+1)$

                    =$\sum\limits_{j=k}^{2k-1}(2j+1)$ - (2k+1) + (2(2k)+1) + (2(2k+1)+1)
                    
                    =$3k^2 - 2k - 1 + 4k + 1 + 4k + 2 + 1$
                    
                    =$3k^2 + 6k + 3$
                    
                    =$3(k^2 + 2k + 1)$
                    
                    =$3(k+1)^2$
                    
Therefore, for all positive integers \emph{n}, $\sum\limits_{j=n}^{2n-1}(2j+1)=3n^{2}$\\

\textbf{6. Use mathematical induction to show that n lines in the plane passing through the same point divide the plane into 2n regions.}

\textbf{Answer:}\\


For n = 1, obviously the plane will be divided into 2n = 2 regions. So P(1) satisfies.

Suppose for n = k, n lines in the plane passing through the same point divide the plane into 2k regions,

For n = k + 1, we can see the last line as the extra line added to the k lines on the plane scenario. 

That is, there are 2k regions before adding the last line.

As all lines are passing through the same point, we can see the last line must be between two existing lines and divide the existing two regions made by the two lines with 2 extra regions.

That is to say, for n = k + 1, P(k+1) = P(k) + 2 = 2k + 2 = 2(k+1)

Therefore, P(k+1) satisfies when P(k) satisfies.

Therefore, for n lines in the plane passing through the same point divide the plane into 2n regions.\\

\textbf{7. Let $a_{1}$=2, $a_{2}$=9, and $a_{n}$=2$a_{n-1}$+3$a_{n-2}$ for n $\ge$ 3. Show that an $\le$ 3n for all positive integers n. }

\textbf{Answer:}\\


For n = 1, a$_{1}$ = 2 $\le$ 3$^{1}$ = 3

For n = 2, a$_{2}$ = 9 $\le$ 3$^{2}$ = 9

For n = 3, 

a$_{3}$ = 2a$_{n-1}$ + 3a$_{n-2}$

\indent\indent                   = 2$a_{2}$ + 3$a_{1}$
                   
\indent\indent                   = 18 + 9
                   
 \indent\indent               = $27 \le 3^3$

Suppose for $n \le k, a_{k} \le 3^k$

So we have a$_{k+1}$ = 2a$_{k}$+3a$_{k-1}$ 

As we have $a_{n} \le 3^n$, we have $a_{k} \le 3^k$ and $a_{k-1} \le 3^(k-1)$

So we have 

$a_{k+1} \le 2 \cdot 3^k + 3 \cdot 3^(k-1)$

 \indent\indent                    =  $2 \cdot 3^k + 3^k$
                     
 \indent\indent                    =  $3^(k+1)$

Therefore when n = k+1, P(n) satisfies.

Therefore for all positive integers n, f(n) $\le$ $3^n$\\

\textbf{8. Find the error in the following proof of this theorem:
Theorem: Every positive integer equals the next largest positive integer.
Proof: Let P(n) be the proposition n=n+1. To show that P(k) = P(k+1), assume that P(k) is true for some k, so that k=k+1. Add 1 to both sides of this equation to obtain k+1=k+2, which is P(k+1). Therefore P(k)=P(k+1) is true. Hence P(n) is true for all positive integers n.}

\textbf{Answer:}\\

Before assuming P(k) is true for some k, so that k=k+1, we should show that for specific starting examples stand first.

For example, for n = 1, left side = 1, right side = 1 + 1

Obviously 1 $\neq$ 2

So we cannot assume that P(k) is true for some k because we even cannot find a specific example to let P(k) is true.

Therefore this proof is wrong.\\

\textbf{9. Give a recursive definition with initial conditions for the function f(n)=5n+2,n=1,2,3,....}

\textbf{Answer:}\\


By observation, 
f(1) = 7
f(2) = 12
f(3) = 17
f(4) = 22

Let us assume f(n) = f(n-1) + 5 for $n \ge 2$

For n = 2, we have f(2) = f(1) + 5 = 7 + 5 = 5*2 + 2 = 12, which satisfy P(n)

Let us assume for n = k, we have f(k) = f(k-1) + 5

So for n = k+1, f(k+1) = 5(k+1) + 2 = 5k + 2 + 5

As f(k) = 5k+2

So we have f(k+1) = f(k) + 5

Therefore, the recursive definition of the function f(n) is

f(n) = 7 for $n = 1$

f(n) = f(n-1) + 5 for $n \ge 2$\\

\textbf{10. Give a recursive definition with initial conditions for {$a_{n}$} where $a_{n}$=$2^{\frac{1}{2^n}}$}

\textbf{Answer:}\\


By observation, 

$a_{1}$ = $2^{\frac {1}{2}}$

$a_{2}$ = $2^{\frac {1}{4}}$

$a_{3}$ = $2^{\frac {1}{8}}$

Let us assume $f(1) = 2^{\frac{1}{2}}, f(n) = f(n-1) ^{\frac {1}{2}} for\ any\ n \ge 2$.

For n = 2, $f(2) = 2^{\frac{1}{4}} = f(1) ^{\frac {1}{2}} = (2^{\frac{1}{2}})^{\frac{1}{2}} = 2^{\frac{1}{4}}$

So for n = 2, P(n) satisfies

Let us assume for n = k P(n) satisfies. So we have f(k) = $2^{\frac{1}{2^k}}$

When n = k+1, we have:

f(k+1) = $2^{\frac{1}{2^(k+1)}} = (2^{\frac{1}{2^k}})^{\frac{1}{2}}$

\indent\indent                             = $f(k)^{\frac{1}{2}}$

So the recursion definition of the function is: 

for f(1) = $2^{\frac{1}{2}}$, $f(n) = f(n-1) ^{\frac {1}{2}}$ for any n $\ge$ 2.\\


\textbf{11. Consider the following program segment:}

i := 1

total := 1 

while i $<$ n

\indent\indent  i := i + 1
  
\indent\indent  total := total + i

\textbf{Let p be the proposition $``total = \frac{i(i+1)}{2}\ and\ i \le n.$'' Use mathematical induction to prove that p is a loop invariant.}

\textbf{Answer:}\\

\noindent Basic Step:

 When n = 1, the program executes like this: 

  i = 1; total = 1; i $<$ n is false hence the while loop does not executes. 
  
  Hence the total = 1
  
  This satisfies the $total = \frac{i(i+1)}{2} = \frac{2}{2} = 1$
  
  This also satisfies i $\le$ n.

Assume p is loop invariant for n = k.

That means after the kth loop execution, the statement hold.\\

\noindent Induction Step:

 n = k, $total = \frac{k(k+1)}{2} and\ k \le n$ is true

Let's assume R(n) to be the 
value of total at the end of the code execution

So when $n=k+1$

We know that the statement is true after the kth loop, when n = k+1, it has one more loop after n = k, 

so at the beginning of the k+1 th loop,

 $total = \frac{k(k+1)}{2}\ and\ i\le k$ 

Then 
From the code snippet, when n = k+1,
$ i = k < k+1$

After execution:

$ i = i+1 = k+1 \le k+1$

$total = \frac{k(k+1)}{2} + (k+1)=\frac{(k+1)(k+2)}{2} = \frac{(k+1)((k+1)+1)}{2}$

The statement hold, which means when n = k+1, the proposition is true.

Than is, the statement is loop invariant


\end{large}

\end{document}


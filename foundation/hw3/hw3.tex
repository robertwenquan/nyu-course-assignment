1. Suppose you wish to use the Principle of Mathematical Induction to prove that
1⋅1!+2⋅2!+3⋅3!+⋅⋅⋅+n⋅n!=(n+1)!−1 for all n≥1.
(a) Write P(1) (b) Write P(5) (c) Write P(k) (d) Write P(k+1) (e) Use the Principle of Mathematical Induction to prove that P(n) is true for all n≥1

2. Use the Principle of Mathematical Induction to prove that n3>n2+3 for all n≥2. 

3. Use the Principle of Mathematical Induction to prove that 1+3+9+27+⋅⋅⋅+3n=3n+1−12 for all n≥0.

4. Use the Principle of Mathematical Induction to prove that any integer amount of postage from 18 cents on up can be made from an infinite supply of 4-cent and 7-cent stamps.

5. Prove that ∑j=n2n−1(2j+1)=3n2 for all positive integers n. 

6. Use mathematical induction to show that n lines in the plane passing through the same point divide the plane into 2n regions.

7. Let a1=2,a2=9, and an=2an−1+3an−2 for n≥3. Show that an≤3n for all positive integers n. 

8. Find the error in the following proof of this “theorem”: “Theorem: Every positive integer equals the next largest positive integer.” “Proof: Let P(n) be the proposition ‘n=n+1’. To show that P(k)→P(k+1), assume that P(k) is true for some k, so that k=k+1. Add 1 to both sides of this equation to obtain k+1=k+2, which is P(k+1). Therefore P(k)→P(k+1) is true. Hence P(n) is true for all positive integers n.”

9. Give a recursive definition with initial conditions for the function f(n)=5n+2,n=1,2,3,....

10. Give a recursive definition with initial conditions for {an} where an=21/2n.

11. Consider the following program segment:

i := 1
total := 1 
while i < n
  i := i + 1
  total := total + i
Let p be the proposition “total =i(i+1)2 and i≤n.” Use mathematical induction to prove that p is a loop invariant.


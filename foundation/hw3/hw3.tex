\documentclass{article}
\usepackage[utf8]{inputenc}
\usepackage{geometry}     
\usepackage{latexsym}
\geometry{left=2.5cm,right=2.5cm,top=2.5cm,bottom=2.5cm}
\usepackage{fancyhdr}
\pagestyle{fancy}
\usepackage{amssymb}
\title{CS6003-INET Foundation of Computer Science Problem Set3}
\author{Robert Wen (robert.wen@nyu.edu), NetID: qw476, N12246277}
\date{September 2015}

\begin{document}
\begin{large}

\maketitle


1. Suppose you wish to use the Principle of Mathematical Induction to prove that
1$\cdot$1!+2$\cdot$2!+3$\cdot$3! + $\cdots$ +n$\cdot$n!=(n+1)! - 1 for all n $\ge$ 1.

(a) Write P(1) (b) Write P(5) (c) Write P(k) (d) Write P(k+1) (e) Use the Principle of Mathematical Induction to prove that P(n) is true for all n

(a) P(1) = 1$\cdot$1! = 1

(b) P(5) = 1$\cdot$1! + 2$\cdot$2! + 3$\cdot$3! + 4$\cdot$4! + 5$\cdot$5!
         = 1$\cdot$1  + 2$\cdot$2  + 3$\cdot$6  + 4$\cdot$24 + 5$\cdot$120
         = 719

(c) P(k) = 1$\cdot$1! + 2$\cdot$2! + 3$\cdot$3! + ... + k$\cdot$k!

(d) P(k+1) = 1$\cdot$1! + 2$\cdot$2! + 3$\cdot$3! + ... + k$\cdot$k! + (k+1)$\cdot$(k+1)!

(e) For n = 1, P(1) = 1$\cdot$1! = 1 = (1+1)! - 1 = 2 - 1 = 1
\indent    Suppose when n = k and k $\ge$ 1, P(k) = 1$\cdot$1! + $\cdots$ + k$\cdot$k! = (k+1)! - 1
\indent    We know P(k+1) = P(k) + (k+1)$\cdot$(k+1)! 
\indent \indent           = (k+1)! - 1 + (k+1)$\cdot$(k+1)!
\indent \indent           = (k+1+1)(k+1)! - 1
\indent \indent           = (k+2)! - 1
\indent \indent           = ((k+1)+1)! - 1
\indent    So for all n where n $\ge$ 1, 1$\cdot$1!+2$\cdot$2!+3$\cdot$3! + $\cdots$ +n$\cdot$n!=(n+1)! - 1

\noindent2. Use the Principle of Mathematical Induction to prove that $n^{3}>n^{2}+3$ for all $n\ge 2$. \\

3. Use the Principle of Mathematical Induction to prove that 1+3+9+27+ $\cdots$ +3n=3n+1−12 for all n $\ge$ 0.

4. Use the Principle of Mathematical Induction to prove that any integer amount of postage from 18 cents on up can be made from an infinite supply of 4-cent and 7-cent stamps.

5. Prove that ∑j=n2n−1(2j+1)=3n2 for all positive integers n. 

6. Use mathematical induction to show that n lines in the plane passing through the same point divide the plane into 2n regions.

7. Let a1=2,a2=9, and an=2an−1+3an−2 for n≥3. Show that an≤3n for all positive integers n. 

8. Find the error in the following proof of this “theorem”: “Theorem: Every positive integer equals the next largest positive integer.” “Proof: Let P(n) be the proposition ‘n=n+1’. To show that P(k)→P(k+1), assume that P(k) is true for some k, so that k=k+1. Add 1 to both sides of this equation to obtain k+1=k+2, which is P(k+1). Therefore P(k)→P(k+1) is true. Hence P(n) is true for all positive integers n.”

9. Give a recursive definition with initial conditions for the function f(n)=5n+2,n=1,2,3,....

10. Give a recursive definition with initial conditions for {an} where an=21/2n.

11. Consider the following program segment:

i := 1
total := 1 
while i < n
  i := i + 1
  total := total + i
Let p be the proposition “total =i(i+1)2 and i≤n.” Use mathematical induction to prove that p is a loop invariant.

\end{large}

\end{document}


\documentclass[11pt, oneside]{article}   	% use "amsart" instead of "article" for AMSLaTeX format
\usepackage{geometry}                		% See geometry.pdf to learn the layout options. There are lots.
\geometry{letterpaper}                   		% ... or a4paper or a5paper or ... 
%\geometry{landscape}                		% Activate for for rotated page geometry
%\usepackage[parfill]{parskip}    		% Activate to begin paragraphs with an empty line rather than an indent
\usepackage{graphicx}				% Use pdf, png, jpg, or eps§ with pdflatex; use eps in DVI mode
\usepackage{latexsym}
								% TeX will automatically convert eps --> pdf in pdflatex	?	
\usepackage{amssymb}
\usepackage[metapost] {mfpic}
\usepackage{fancyhdr}
\pagestyle{fancy}
\lhead{PS 3}
\rhead{Net ID:cc4584}	
\chead{Caicai CHEN}
\usepackage{geometry}
\geometry{left=2.8cm,right=2.8cm,top=2.5cm,bottom=2.5cm}



\title{Problem Set 3}
\author{Caicai CHEN}
\date{September. 12, 2014}							% Activate to display a given date or no date

\begin{document}
\maketitle

\noindent1. Suppose you wish to use the Principle of Mathematical Induction to prove that:

$1\cdot 1!+2\cdot 2! + 3\cdot 3!+\cdot\cdot\cdot+n\cdot n!=(n+1)!-1$ for all $n\ge 1$.\\
(a) Write $P(1)$ (b) Write $P(5)$ (c) Write $P(k)$ (d) Write $P(k+1)$ (e) Use the Principle of Mathematical Induction to prove that $P(n)$ is true for all $n\ge 1$\\
\textbf{\emph{Solution:}}

(a) $P(1):\ 1\cdot 1!=(1+1)!-1$

(b) $P(5):\ 1\cdot 1!+2\cdot 2!+3\cdot 3!+4\cdot 4!+5\cdot 5!=(5+1)!-1$

(c) $P(k):\ 1\cdot 1!+2\cdot 2! + 3\cdot 3!+\cdot\cdot\cdot+k\cdot k!=(k+1)!-1$

(d) $P(k+1):\ 1\cdot 1!+2\cdot 2! + 3\cdot 3!+\cdot\cdot\cdot+(k+1)\cdot (k+1)!=((k+1)+1)!-1$

(e) (i) Obviously, when $n=1$,  $1\cdot 1!=1=(1+1)!-1$ is true.

\quad (ii)Assume that $P(k)$ is true for an arbitrary positive integer \emph{k}, then:

when $n=k+1$:

$1\cdot 1!+2\cdot 2! + 3\cdot 3!+\cdot\cdot\cdot+(k+1)\cdot (k+1)!=1\cdot 1!+2\cdot 2! + 3\cdot 3!+\cdot\cdot\cdot+k\cdot k!+(k+1)\cdot (k+1)!$

$=(k+1)!-1+(k+1)\cdot (k+1)!$

$=(k+2)\cdot(k+1)!-1$

$=(k+2)!-1$

$=((k+1)+1)!-1$

which means $P(k+1)$ is also true.

So according to Principle of Mathematical Induction, $P(n)$ is true for all $n\ge 1$.\\

\noindent2. Use the Principle of Mathematical Induction to prove that $n^{3}>n^{2}+3$ for all $n\ge 2$. \\
\textbf{\emph{Solution:}}

(1) When $n=2$ : $n^{3}=2^{3}=8>2^{2}+3=7$ is true.

(2) Assume that $P(k)$ is true for an arbitrary positive integer \emph{k} (except 1), then:

when $n=k+1$:

\begin{tabular}{@{} rcl @{}}
$(k+1)^{3}$&=&$k^{3}+3k^{2}+3k+1$\\
&$>$&$(k^{2}+3)+3k^{2}+3k+1$\\
&=&$4k^{2}+3k+4$\\
&$>$&$k^{2}+2k+4$ \quad $(\because k>0, \therefore k^{2}>0, 3k^{2}+k>0)$\\
&=&$(k+1)^{2}+3$
\end{tabular}

which means $P(k+1)$ is also true.

So according to Principle of Mathematical Induction, $P(n)$ is true for all $n\ge 2$.\\

\noindent3. Use the Principle of Mathematical Induction to prove that $1+3+9+27+\cdot\cdot\cdot+3^{n}=\frac{3^{n+1}-1}{2}$ for all $n\ge 0$.\\
\textbf{\emph{Solution:}}

(1) When $n=0$: $1=\frac{3^{0+1}-1}{2}$ is true.

(2) Assume that $P(k)$ is true for an arbitrary nonnegative integer \emph{k}, then:

when $n=k+1$:

$1+3+9+27+\cdot\cdot\cdot+3^{k}+3^{k+1}$

=$\frac{3^{k+1}-1}{2}\ +\ 3^{k+1}$

=$\frac{3\cdot 3^{k+1}-1}{2}$

=$\frac{3^{(k+1)+1}-1}{2}$

which means $P(k+1)$ is also true.

So according to Principle of Mathematical Induction, $P(n)$ is true for all $n\ge 0$.\\

\noindent4. Use the Principle of Mathematical Induction to prove that any integer amount of postage from 18 cents on up can be made from an infinite supply of 4-cent and 7-cent stamps.\\
\textbf{\emph{Solution:}} That's translate the sentence into logic expression:

$\forall n\ge 18,\ n=4a+7b\ (a,b\in N)$  has solution.

(1) When $n=18$, $a=1\ and\ b=2$ is a solution to equation $18=4a+7b$

that means P(18) is true.

(2) Assume that $P(k)$ is true for $k\ge18$, that is

\indent \qquad $k=4a+7b$ has a solution

Let's consider the situation of $n=k+1$

\qquad $k+1=4a+7b+1$

when $b>0$, $(b-1)\in N$

\qquad $k+1=4a+7(b-1)+7+1=4(a+2)+7(b-1)$

when $b=0$, we can know that $4a\ge18$ and $a\in N$, so $a\ge 5$

\qquad $k+1=4a+7b+1=4(a-5)+20+7b+1=4(a-5)+7(b+3)$

From above, we can say that $P(k+1)$ is true.

So according to Principle of Mathematical Induction, $P(n)$ is true for all $n\ge 18$. Which means any integer amount of postage from 18 cents on up can be made from an infinite supply of 4-cent and 7-cent stamps.\\

\noindent5. Prove that $\sum\limits_{j=n}^{2n-1}(2j+1)=3n^{2}$ for all positive integers \emph{n}.\\
\textbf{\emph{Solution:}}

(1) When n=1, $P(1)$ is true since $\sum\limits_{1}^{1}(2j+1)=2+1=3=3\times 1^{2}$

(2) Assume that $P(k)$ is true for an arbitrary positive integer \emph{k}, that is $\sum\limits_{j=k}^{2k-1}(2j+1)=3k^{2}$

when $n=k+1$:

$\sum\limits_{j=k+1}^{2(k+1)-1}(2j+1)=\sum\limits_{j=k+1}^{2k+1}(2j+1)=-\sum\limits_{j=k}^{k}(2j+1)+\sum\limits_{j=k}^{2k-1}(2j+1)+\sum\limits_{j=2k}^{2k+1}(2j+1)$

=$\sum\limits_{j=k}^{2k-1}(2j+1)+(2\times 2k+1)+(2\times(2k+1)+1)-(2k+1)$

=$3k^{2}+3\times 2k+ 3\times 1$

=$3(k+1)^{2}$

so we have proved that $P(k+1)$ is true.

According to Principle of Mathematical Induction, $P(n)$ is true for all positive integers.\\

\noindent6. Use mathematical induction to show that \emph{n} lines in the plane passing through the same point divide the plane into \emph{2n} regions.\\
\textbf{\emph{Solution:}}Let's create a $xOy$ coordinate system. Let the point that every line pass is point O, and set the last line be y-axis, we order all the line from y-axis in clockwise direction. 

%\opengraphsfile{myfig}
%\setlength{\mfpicunit}{1cm}
%\begin{mfpic} {-2.5}{2.5}{-2.5}{2.5}
%\drawcolor{blue}
%\function{-2,2,.1}{3\times x}
%\function{-2,2,.1}{1\times x}
%\function{-2,2,.1}{\frac{1}{4}\times x}
%\function{-2,2,.1}{($-1$)\times x}
%\function{-2,2,.1}{$-5$\times x}
%\end{mfpic}

(1) When n=1, there is only one line, that's y-axis, it divide the plane into 2 regions. So $P(1)$ is true.

(2) Assume that $P(k)$ is true for an arbitrary positive integer \emph{k}.

When $n=k+1$, it means that we add a line, which is between the No.2 line and No.(k+1) line of the xOy coordinate system of $P(k+1)$, while that two line are adjacent in the xOy of $P(k)$.

As there were two separate regions between this two line in xOy of $P(k)$, and the newest line divides each of the two parts into two parts.

So there will add two more regions in the xOy coordinate system of $P(k)$.

So there will be $2k+2=2(k+1)$ regions in the xOy coordinate system of $P(k+1)$, which means $P(k+1)$ is true.

According to Principle of Mathematical Induction, $P(n)$ is true for all positive integers.\\

\noindent7. Let $a_{1}=1,a_{2}=9$ and $a_{n}=2a_{n-1}+3a_{n-2}$ for $n\ge 3$. Show that $a_{n}\le 3^{n}$ for all positive integers \emph{n}.\\
\textbf{\emph{Solution:}}

(1) When $n=1$, $a_{1}=1\le 3^{1}$, $P(1)$ is true.

(2) When $n=2$, $a_{2}=9 \le 3^{2}=9$, $P(2)$ is true.

(3) Assume, for an arbitrary positive integer \emph{k} and $k\ge 3$, $P(k)$ is true and $P(k-1)$ is true.

When $n=k+1$,

\begin{tabular}{@{} rcl @{}}
\qquad $a_{k+1}$&=&$2a_{k}+3a_{k-1}$\\
&$\le$&$2\times 3^{k}+3\times 3^{k-1}$\\
&=&$6\times 3^{k-1}+3\times 3^{k-1}$\\
&=&$9\times3^{k-1}$\\
&=&$3^{k+1}$
\end{tabular}

So we have proved that $P(k+1)$ is true.

According to Priciple of Mathematical Induction, $P(n)$ is true for all positive integers.\\

\noindent8. Find the error in the following proof of this ``theorem'': ``Theorem: Every positive integer equals the next largest positive integer.'' ``Proof: Let $P(n)$ be the proposition `n=n+1'. To show that $P(k)\to P(k+1)$, assume that $P(k)$ is true for some $k$, so that $k=k+1$. Add 1 to both sides of this equation to obtain $k+1=k+2$, which is $P(k+1)$. Therefore $P(k)\to P(k+1)$ is true. Hence $P(n)$ is true for all positive integers $n$.''\\
\textbf{\emph{Solution:}} This Proof tries to use Principle of Mathematical Induction, but it lacks the first step of the Principle.

We can see that when $n=1$, $1=1+1$ is False.

So this proof is wrong.\\

\noindent9. Give a recursive definition with initial conditions for the function $f(n)=5n+2, n=1,2,3,...$\\
\textbf{\emph{Solution:}}

The first part of the recursive definition is 

\qquad $f(0)=5\times 0 + 2 = 2$

The second part is :

\qquad $f(n+1)=5(n+1)+2=(5n+2)+5=f(n)+5$\\

\noindent10. Give a recursive definition with initial conditions for $\{a_{n}\}$ where $a_{n}=2^{1/2^{n}}$.\\
\textbf{\emph{Solution:}}

The first part of the recursive definition is 

\qquad $a_{0}=2^{1/1}=2$

The second part is:

\qquad $a_{n+1}=2^{1/2^{(n+1)}}=2^{(1/2^{n})\times (1/2)}=(2^{1/2^{n}})^{1/2}=(a_{n})^{1/2}$\\

\noindent11. Consider the following program segment:

$\quad i\ :=1$

$\quad total\ :=1$

$\quad while\ i<n$

$\quad \qquad i\ :=i+1$

$\quad \qquad total\ :=total+i$

Let \emph{p} be the proposition ``$total=\frac{i\cdot (i+1)}{2}$ and $i\le n$.'' Use mathematical induction to prove that \emph{p} is a loop invariant.\\
\textbf{\emph{Solution:}}

(1) When $n=1$, obviously, it's true. 

(2) Assume that \emph{p} is loop invariant for $n=k, k\in N^{+}$

That means, if \emph{p} is true before the program segment is executed, \emph{p} and $\neg$condition are true after termination.

 \emph{p} is true after a new execution of the \textbf{while} loop when $i<k$

So when $n=k+1$, we just need to prove that when $i=k$, \emph{p} is still true after a new execution.

At the beginning of this loop, $total=\frac{k\cdot (k+1)}{2}$ and $k< n$, the condition holds.

The \textbf{while} loop execute again.

\qquad \qquad $i=k+1$ where $i\le n$ as $n=k+1$

$\therefore \ total=\frac{k\cdot (k+1)}{2}+(k+1)=\frac{(k+1)\cdot ((k+1)+1)}{2}$

Hense \emph{p} is still true for a new execution.

According to the Principle of Mathematical Induction, \emph{p} is a loop invariant.

\end{document}

\documentclass{article}
\usepackage[utf8]{inputenc}

\title{CS6003-INET Foundation of Computer Science Problem Set1}
\author{Robert Wen (robert.wen@nyu.edu), NetID: qw476, N12246277}
\date{September 2015}

\begin{document}

\maketitle
\setlength{\parindent}{2ex}
\textbf{1. Find a proposition with three variables p,q, and r that is true when p and r are true and q is false, and false otherwise.}

\textbf{Answer:} $p \land r \land \lnot q$
\newline

\textbf{2. Determine whether $(p \to q) \land (\lnot p \to q) \equiv q$.}

Answer: xxx
\newline

\textbf{3. Determine whether this proposition is a tautology: $((p \to q) \land \lnot p) \to \lnot q$.}

\textbf{4. Using c for "it is cold" and w for "it is windy", write "To be windy it is necessary that it be cold" in symbols.}

\textbf{5. Determine whether the following compound proposition is satisfiable $(p \to q) \land (q \to \lnot p)\land (p \lor q)$}

In the following 5 problems P(x,y) means "x+2y=xy", where x and y are integers. Determine the truth value of the statement.

\textbf{6. $\forall x \exists yP(x,y)$.}

\textbf{7. $\exists x \forall yP(x,y)$.}

\textbf{8. $\forall y \exists xP(x,y)$.}

\textbf{9. $\exists y \forall xP(x,y)$.}

\textbf{10. $\lnot \forall x \exists y \lnot P(x,y)$.}

\textbf{11. Determine whether the following argument is valid:}

\textbf{$p \to r$}

\textbf{$q \to r$}

\textbf{$\lnot (p \lor q)$}

----------------

$\therefore \lnot r$

12. Show that the premises "Everyone who read the textbook passed the exam", and "Ed read the textbook" imply the conclusion "Ed passed the exam".

13. Determine whether the premises "No juniors left campus for the weekend" and "Some math majors are not juniors" imply the conclusion "Some math majors left campus for the weekend."

14. Show that the premise "My daughter visited Europe last week" implies the conclusion "Someone visited Europe last week".

15. Give a proof by contradiction of the following: “If n is an odd integer, then n2 is odd”.

16. Consider the following theorem: If x is an odd integer, then x+2 is odd. Give a direct proof of this theorem

17. Consider the following theorem: If x is an odd integer, then x+2 is odd. Give a proof by contraposition of this theorem.

18. Consider the following theorem: If x is an odd integer, then x+2 is odd. Give a proof by contradiction of this theorem.

19. Prove: if m and n are even integers, then mn is a multiple of 4.

20. Prove or disprove: For all real numbers x and y, $\lfloor x-y\rfloor =\lfloor x \rfloor -\lfloor y \rfloor$.

21. Prove that the following three statements about positive integers n are equivalent: (a) n is even; (b) n3+1 is odd; (c) n2-1 is odd.

22. Given any 40 people, prove that at least four of them were born in the same month of the year.

23. Prove that the equation $2x2+y2=14$ has no positive integer solutions.

\end{document}


\documentclass{article}
\usepackage[utf8]{inputenc}
\usepackage{geometry}     
\usepackage{latexsym}
\geometry{left=2.5cm,right=2.5cm,top=2.5cm,bottom=2.5cm}
\usepackage{fancyhdr}
\pagestyle{fancy}
\usepackage{amssymb}
\title{CS6003-INET Foundation of Computer Science Problem Set2}
\author{Robert Wen (robert.wen@nyu.edu), NetID: qw476, N12246277}
\date{September 2015}

\begin{document}
\begin{large}

\maketitle



\textbf{1.Prove that \emph{$\overline{A\cap B}$=$\overline{A} \cup \overline{B}$} by giving a proof using logical equivalence.}

\textbf{Answer:} \\




\textbf{In Problems 2-4 mark each statement TRUE or FALSE. Assume that the statement applies to all sets.} \\

\textbf{2. A-(B-C)=(A-B)-C.}

\textbf{Answer:} \\





\textbf{3.(A-C)-(B-C)=A-B}.

\textbf{Answer:} \\





\textbf{4. $\overline{A\cup \overline{B}} \cup \overline{A}=\overline{A}$}

\textbf{Answer:} \\




\textbf{5. Find $\bigcap\limits_{i=1}^{+\infty}[1-\frac{1}{i}, 1]$.}

\textbf{Answer:} \\




\textbf{6. Suppose $A=\{x,y\}$ and $B=\{x,\{x\}\}$. Mark the statement TRUE or FALSE: $\{x\}\subseteq A-B$.}

\textbf{Answer:} \\





\textbf{7. Suppose $A=\{1,2,3,4,5\}$. Mark the statement TRUE or FALSE:}

\textbf{Answer:} \\




\textbf{8. Determine whether the set is finite or infinite. If the set is finite, find its size:\\
\indent$\{x|x\in N\ and\ 4{x^{2}}-8=0\}$.}

\textbf{Answer:} \\




\textbf{9. Determine whether the following set is countable or uncountable. If it is countably infinite, exhibit a one-to-one correspondence between the set of positive integers and the set:\\
The set of irrational numbers between $\sqrt{2}$ and $\frac{\pi}{2}$.}

\textbf{Answer:} \\





\textbf{10. Show that (0,1) has the same cardinality as (0,2).}

\textbf{Answer:} \\




\textbf{11. Suppose $f:N\to N$ has the rule $f(n)=4n+1$. Determine whether $f$ is 1-1.}

\textbf{Answer:} \\




\textbf{12. Suppose $f:N\to N$ has the rule $f(n)=4n+1$. Determine whether \emph{f} is onto \emph{N}.}

\textbf{Answer:} \\




\textbf{13. Suppose $f:Z\to Z$ has the rule $f(n)=3{n^{2}}-1$. Determine whether \emph{f} is 1-1.}

\textbf{Answer:} \\




\textbf{14. Suppose $f:Z\to Z$ has the rule $f(n)=3n-1$. Determine whether \emph{f} is onto \emph{Z}.}

\textbf{Answer:} \\




\textbf{15. Suppose $f:N\to N$ has the rule $f(n)=3{n^{2}}-1$. Determine whether f is 1-1.}

\textbf{Answer:} \\




\textbf{16. Suppose $f:N\to N$ has the rule $f(n)=4{n^{2}-1}$. Determine whether \emph{f} is onto \emph{N}.}

\textbf{Answer:} \\





\textbf{17. Suppose $g:R\to R$ where $g(x)=\lfloor \frac{x-1}{2}\rfloor$.}

\textbf{(i)Is \emph{g} 1-1?}

\textbf{(ii)Is \emph{g} onto \emph{R}}

\textbf{Answer:} \\




\textbf{18. Find $f\circ g$.} 


\textbf{Answer:} \\




\textbf{19. Find $f^{-1}$.}

\textbf{Answer:} \\




\textbf{20. Suppose $g:A\to B$ and $f:B\to C$, where $f\circ g$ is 1-1 and \emph{g} is 1-1. Must \emph{f} be 1-1?}

\textbf{Answer:} \\





\textbf{21. Suppose $g:A\to B$ and $f:B\to C$, where $f\circ g$ is 1-1 and \emph{f} is 1-1. Must g be 1-1?}

\textbf{Answer:} \\




\textbf{22. Verify that $a_{n}={3^{n}+1}$ is a solution to the recurrence relation $a_{n}=4a_{n-1}-3a_{n-2}$.}

\textbf{Answer:} \\





\textbf{23. You take a job that pays \$25,000 annually.}

\textbf{(a) How much do you earn \emph{n} years from now if you receive a three percent raise each year?}

\textbf{(b) How much do you earn \emph{n} years from now if you receive a five percent raise each year? }

\textbf{(c) How much do you earn \emph{n} years from now if each year you receive a raise of \$1000 plus two percent of your previous year's salary.}

\textbf{Answer:} \\













\end{large}






\end{document}
\documentclass{article}
\usepackage[utf8]{inputenc}
\usepackage{geometry}     
\usepackage{latexsym}
\geometry{left=2.5cm,right=2.5cm,top=2.5cm,bottom=2.5cm}
\usepackage{fancyhdr}
\pagestyle{fancy}
\usepackage{amssymb}
\title{CS6003-INET Foundation of Computer Science Problem Set2}
\author{Robert Wen (robert.wen@nyu.edu), NetID: qw476, N12246277}
\date{September 2015}

\begin{document}
\begin{large}

\maketitle



\textbf{1.Prove that \emph{$\overline{A\cap B}$=$\overline{A} \cup \overline{B}$} by giving a proof using logical equivalence.}

\textbf{Answer:} \\

\indent\indent\indent		     $\overline{A\cap B}$=${x|x \notin A\cap B}$

 \indent\indent \indent                  =${x|\lnot(x \in A\cap B)}$
                    
 \indent\indent \indent                  =${x|\lnot(x \in A \land x \in B)}$
                    
 \indent\indent \indent                  =${x|(x \notin A) \lor (x \notin B)}$
                    
  \indent\indent \indent                 =${x|(x \in \overline{A}) \lor (x \in \overline{B})}$
                    
 \indent\indent \indent                  =${x|x \in \overline{A} \cup \overline{B}}$
                    
 \indent\indent  \indent                 =$\overline{A} \cup \overline{B}$\\

\textbf{In Problems 2-4 mark each statement TRUE or FALSE. Assume that the statement applies to all sets.} \\

\textbf{2. A-(B-C)=(A-B)-C.}

\textbf{Answer:} \\

A-(B-C) = $A-(B \cap \overline{C})$

        = $A \cap \overline{(B \cap \overline{C})}$
        
        = $A \cap (\overline{B} \cup C)$\\

(A-B)-C = $A \cap \overline{B} \cap \overline{C}$

        = $A \cap (\overline{B} \cap \overline{C})$\\

As $\overline{B} \cup C$ does not equal to $\overline{B} \cap \overline{C}$
$\therefore A-(B-C) \neq (A-B)-C$\\

\textbf{3.(A-C)-(B-C)=A-B}.

\textbf{Answer:} \\

$(A-C)-(B-C) = (A \cap \overline{C}) - (B \cap \overline{C})$

            $= (A \cap \overline{C}) \cap \overline{(B \cap \overline{C})}$
            
            $= (A \cap \overline{C}) \cap (\overline{B} \cup C)$
            
            $= ((A \cap \overline{C}) \cap \overline{B}) \cup ((A \cap \overline{C}) \cap C)$
            
            $= (A \cap \overline{C} \cap \overline{B}) \cup \emptyset$
            
            $= A \cap \overline{C} \cap \overline{B}$
            
            $= A \cap \overline{B} \cap \overline{C}$\\

$A-B = A \cap \overline{B}$\\

As $A \cap \overline{B} \neq A \cap \overline{B} \cap \overline{C}$

$\therefore (A-C)-(B-C) \neq A-B$\\

\textbf{4. $\overline{A\cup \overline{B}} \cup \overline{A}=\overline{A}$}

\textbf{Answer:} \\

As $(X \cap Y) \cup X = X$

$\overline{A\cup \overline{B}} \cup \overline{A} = (\overline{A} \cap B) \cup \overline{A}$
                                                 = $\overline{A}$

$\therefore $the equality is true.\\

\textbf{5. Find $\bigcap\limits_{i=1}^{+\infty}[1-\frac{1}{i}, 1]$.}

\textbf{Answer:} \\

             CAICAI
For any$ i, j \in [1, +\infty),$ i and j are integers and i < j
                         CAICAI
we know $1 - \frac{1}{i} \le \frac{1}{j}$

we know $[1 - \frac{1}{i}, 1] \cap [1 - \frac{1}{j},1] = [1 - \frac{1}{j},1]$

Therefore, $\bigcap\limits_{i=1}^{+\infty}[1-\frac{1}{i}, 1]$ = $\lim_{i\to + \infty} [1 - \frac{1}{i},1]$
                                                              = $[1 - \frac{1}{\lim_{i\to + \infty}i}, 1]$
                                                              = $[1 - 0, 1]$
                                                              = 1\\


\textbf{6. Suppose $A=\{x,y\}$ and $B=\{x,\{x\}\}$. Mark the statement TRUE or FALSE: $\{x\}\subseteq A-B$.}

\textbf{Answer:} \\

$A-B = \{x|x \in A \cap \overline B\}$

    $= \{x|x \in A \land x \notin B\}$
    
   $ = \{y\}$

As$ \{x\} \subsetneq \{y\}$

$\therefore \{x\} \subseteq A-B\ is\ FALSE$


\textbf{7. Suppose $A=\{1,2,3,4,5\}$. Mark the statement TRUE or FALSE:}

{{3}} $\subseteq$ P(A).

\textbf{Answer:} \\

As from the definition of P(A), 
        CAICAI
P(A) = $\{\emptyset, \{1\}, \{2\}, \{3\}, \{4\}, \{5\}, \{1,2\}, \{1,3\}, ..., \{1,2,3\}, \{1,2,4\}, ..., \{1,2,3,4\}, \{1,2,3,5\}, ...,\{1,2,3,4,5\}\}$
$\therefore {{3}} \subseteq P(A)$


\textbf{8. Determine whether the set is finite or infinite. If the set is finite, find its size:\\
\indent$\{x|x\in N\ and\ 4{x^{2}}-8=0\}$.}

\textbf{Answer:} \\
                                              CAICAI
As the only solution to $4{x^{2}}-8=0\}$ are $\sqrt 2$ and -$\sqrt 2$, 
                                       CAICAI
So $\{x|x\in N\ and\ 4{x^{2}}-8=0\} = \emptyset $
Therefore the set is finite and the size of the set is zero.


\textbf{9. Determine whether the following set is countable or uncountable. If it is countably infinite, exhibit a one-to-one correspondence between the set of positive integers and the set:\\
The set of irrational numbers between $\sqrt{2}$ and $\frac{\pi}{2}$.}

\textbf{Answer:} \\

Let A is the set of the real numbers between $\sqrt{2}$ and $\frac{\pi}{2}$
Let B is the set of the rational numbers between $\sqrt{2}$ and $\frac{\pi}{2}$
Let C is the set of the irrational numbers between $\sqrt{2}$ and $\frac{\pi}{2}$

We know all real numbers are uncountable, and real numbers between $\sqrt{2}$ and $\frac{\pi}{2}$ are uncountable.
We also know rational numbers are countable and rational numbers between $\sqrt{2}$ and $\frac{\pi}{2}$ are countable.

Assume irrational numbers between $\sqrt{2}$ and $\frac{\pi}{2}$ are countable, 
because A = B $\cup$ C, and from theorm we know ``If A and B are countable sets, then A $\cup$ B is also countable''
So A will be countable if B and C are both countable, with contradicts with that A is uncountable.

Therefore the assumption is false. 
$\therefore$ The set of irrational numbers between $\sqrt{2}$ and $\frac{\pi}{2}$ is uncountable.\\


\textbf{10. Show that (0,1) has the same cardinality as (0,2).}

\textbf{Answer:} \\



\textbf{11. Suppose $f:N\to N$ has the rule $f(n)=4n+1$. Determine whether $f$ is 1-1.}

\textbf{Answer:} \\

In order to prove f is one-to-one, we need to prove for any a, b, and a $\neq$ b, f(a) $\neq$ f(b)

since $f(b) - f(a) = (4b + 1) - (4a + 1) = 4(b-a) \neq 0$

Therefore we have f is 1-1.\\


\textbf{12. Suppose $f:N\to N$ has the rule $f(n)=4n+1$. Determine whether \emph{f} is onto \emph{N}.}

\textbf{Answer:} \\

Function f is onto if for any y there is an x to let f(x) = 4x + 1 = y

For this function, when y = 4, we cannot find an x that is an natural number to satisfy 4x + 1 = 4

The solution to this function is $x = \frac{3}{4}$ which is not a natural number.

So function \emph{f} is no onto \emph{N}.\\

\textbf{13. Suppose $f:Z\to Z$ has the rule $f(n)=3{n^{2}}-1$. Determine whether \emph{f} is 1-1.}

\textbf{Answer:} \\

In order to prove f is one-to-one, we need to prove for any a, b, and a $\neq$ b, f(a) $\neq$ f(b)

$f(b) - f(a) = (3b^2 - 1) - (3a^2 - 1) = 3(b^2 - a^2) = 3(b + a)(b - a)$

Even for b $\neq$ a, when b = -a we will have f(b) - f(a) = 0

That means when b $\neq$ a $\land$ b = -a, f(b) = f(a) that contradicts with the condition to be 1-to-1
Therefore f is not 1-1.\\


\textbf{14. Suppose $f:Z\to Z$ has the rule $f(n)=3n-1$. Determine whether \emph{f} is onto \emph{Z}.}

\textbf{Answer:} \\

Function f is onto if $\forall y \exists x (f(x) = y)$

For f(x) = y = 3x - 1, 

$x = \frac{y+1}{3}$

For y = 1, x = $\frac{2}{3}$ which is not an integer. So there is not an x $\in$ Z that satisfies y = 3x - 1, when y = 1

Thus it is false that $\forall y \exists x (f(x) = y)$

Therefore function $\emph{f}$ is not onto $\emph{Z}$\\


\textbf{15. Suppose $f:N\to N$ has the rule $f(n)=3{n^{2}}-1$. Determine whether f is 1-1.}

\textbf{Answer:} \\

In order to prove f is one-to-one, we need to prove for any a, b, and a $\neq$ b, f(a) $\neq$ f(b)

$f(b) - f(a) = (3b^2 - 1) - (3a^2 - 1) = 3(b^2 - a^2) = 3(b + a)(b - a)$

As only b = a or b = -a could lead to f(b) - f(a) = 0

And for the domain of N, $b \neq -a$

So for $b \neq a$, f(b) - f(a) $\neq$ 0

Therefore, f is 1-1


\textbf{16. Suppose $f:N\to N$ has the rule $f(n)=4n^2-1$. Determine whether \emph{f} is onto \emph{N}.}

\textbf{Answer:} \\

Function f is onto if $\forall y \exists x (f(x) = y)$
For $f(x) = y = 4x^2 - 1$
When y = 2, $x = \frac{\sqrt 3}{2}$ which is not a natural number.
So for y = 2, we cannot find a natural number x to satisfy $f(x) = 4x^2 - 1$
Therefore function $\emph{f}$ is not onto $\emph{N}$


\textbf{17. Suppose $g:R\to R$ where $g(x)=\lfloor \frac{x-1}{2}\rfloor$.}

\textbf{(i)Is \emph{g} 1-1?}

\textbf{(ii)Is \emph{g} onto \emph{R}}

\textbf{Answer:} \\

In order to make the \emph{g} 1-1, we need to prove $\forall x, y \in R, x \neq y, f(x) \neq f(y)$
Assume $a, b \in R, and a \neq b,$ we have
$f(b) - f(a) = \lfloor \frac{b-1}{2} \rfloor - \lfloor \frac{a-1}{2} \rfloor$
When b = 5.2 and a = 5.1, 
$f(b) - f(a) = \lfloor \frac{b-1}{2} \rfloor - \lfloor \frac{a-1}{2} \rfloor$
            $= \lfloor \frac{4.2}{2} \rfloor - \lfloor \frac{4.1}{2} \rfloor$
            $= \lfloor 2.1 \rfloor - \lfloor 2.05 \rfloor$
            $= 2 - 2$
            $= 0$
So for $b \neq a$, we have $f(b) = f(a)$
Therefore \emph{g} is not 1-1

In order to make the \emph{g} onto \emph{R}, 
for $f(x) = y = \lfloor \frac{x-1}{2}\rfloor$, we need to prove for $\forall y \exists x$ that satisfies $f(x) = \lfloor \frac{x-1}{2}\rfloor$
Obviously, when y is any non integer real numbers, we cannot find a real number x to satisfy the function because the floor of a real number is always an integer.
Thus it is false that $\forall y \exists x$ that satisfies $f(x) = \lfloor \frac{x-1}{2}\rfloor$
Therefore \emph{g} is not onto \emph{R}.


In questions 18–19 suppose g:A $\to$ B and f:B $\to$ C where A={a,b,c,d},B={1,2,3},C={2,3,6,8}, and g and f are defined by g={(a,2),(b,1),(c,3),(d,2)} and f={(1,8),(2,3),(3,2)}.

\textbf{18. Find $f\circ g$.}

\textbf{Answer:} \\

$f \circ g$ = f(g(x))
$\therefore$ $f \circ g(a) = f(g(a)) = f(2) = 3$
             $f \circ g(b) = f(g(b)) = f(1) = 8$
             $f \circ g(c) = f(g(c)) = f(3) = 2$
             $f \circ g(d) = f(g(d)) = f(2) = 3$
Therefore $f \circ g$ = {(a,3), (b,8), (c,2), (d,3)}


\textbf{19. Find $f^{-1}$.}

\textbf{Answer:} \\

From $f=\{(1,8),(2,3),(3,2)\}$, we can simply say f is 1-to-1 because for any $x_{1}, x_{2} \in \{1,2,3\} $and $x_{1} \neq x_{2}, f(x_{1}) \neq f(x_{2})$
As f:B $\to$ C where B={1,2,3},C={2,3,6,8}
For each $x \in B$, we can find corresponding $y \in C$, but for $6 \in C$, we cannot find corresponding $x \in B$
Hence we cannot have $f^{-1}$ for f:B $\to$ C where $B=\{1,2,3\},C=\{2,3,6,8\} and f=\{(1,8),(2,3),(3,2)\}$

If instead $C = \{2,3,8\}, f^{-1} = \{(8,1),(3,2),(2,3)\}$


\textbf{20. Suppose $g:A\to B$ and $f:B\to C$, where $f\circ g$ is 1-1 and \emph{g} is 1-1. Must \emph{f} be 1-1?}

\textbf{Answer:} \\

Let us have $A = \{1,2\}, B = \{2,4,6\}, C = \{7,8\}$
$g = \{(1,2), (2,4)\}, f = \{(2,7), (4,8), (6,8)\}$
$f\circ g = f(g(x)) = \{(1,7), (2,8)\}$
Obviously $f\circ g$ is 1-1 and \emph{g} is 1-1,
But \emph{f} is not 1-1 because both 4 and 6 $\in$ B map to 8 $\in$ C

Therefore \emph{f} is not 1-1


\textbf{21. Suppose $g:A\to B$ and $f:B\to C$, where $f\circ g$ is 1-1 and \emph{f} is 1-1. Must g be 1-1?}

\textbf{Answer:} \\

Assume g is not 1-1. That means there exists a, b $\in$ A, a $\neq$ b, that satisfies g(a) = g(b)
As \emph{f} is 1-1, that is for any x $\in$ B there will be a value mapped to y $\in$ C
So for $f\circ g$ = f(g(x)), with a, b $\in$ A, and a $\neq$ b, f(g(a)) = f(g(b)), which contracts with that $f\circ g$ is 1-1
because when $f\circ g$ is 1-1, for any a, b $\in$ A, f(g(a)) $\neq$ f(g(b))

So the assumption that g is not 1-1 is false
Therefore g is 1-1


\textbf{22. Verify that $a_{n}={3^{n}+1}$ is a solution to the recurrence relation $a_{n}=4a_{n-1}-3a_{n-2}$.}

\textbf{Answer:} \\

When n > 2
$4a_{n-1}-3a_{n-2} = 4 (3^{n-1} + 1) - 3 (3^{n-2} + 1)$
                    $= (4 - 1) 3^{n-1} + 1$
                    $= 3^{n} + 1$
                    $= a_{n}$

$\therefore$ $a_{n}={3^{n}+1}$ is a solution to the recurrence relation $a_{n}=4a_{n-1}-3a_{n-2}$.


\textbf{23. You take a job that pays \$25,000 annually.}

\textbf{(a) How much do you earn \emph{n} years from now if you receive a three percent raise each year?}

\textbf{(b) How much do you earn \emph{n} years from now if you receive a five percent raise each year? }

\textbf{(c) How much do you earn \emph{n} years from now if each year you receive a raise of \$1000 plus two percent of your previous year's salary.}

\textbf{Answer:} \\

(a) for 1 year after now, the salary will be $25000 * 1.03$
    for 2 years after now, the salary will be $25000 * 1.03 * 1.03 = 25000 * 1.03^2$
    for i years after now, the salary will be $25000 * 1.03^i$
    for n years from now, the salary will be $25000 * 1.03^n$

    So $S_{n} = \frac{25000 (1 - 1.03^(n+1))}{1-1.03}$

(b) Likewise, 
    for 1 year after now, the salary will be $25000 * 1.05$
    for 2 years after now, the salary will be $25000 * 1.05 * 1.05 = 25000 * 1.05^2$
    for i years after now, the salary will be $25000 * 1.05^i$
    for n years from now, the salary will be $25000 * 1.05^n$

    So $S_{n} = \frac{25000 (1 - 1.05^(n+1))}{1-1.05}$

(c) for 1 year after now, the salary will be $25000 + 1000 + 0.02 * 25000$
    for 2 years after now, the salary will be $(25000 * 1.02 + 1000) * 1.02 + 1000$
                                             = $25000 * 1.02^2 + 1000 * 1.02 + 1000$
    for 3 years from now, the salary will be $(25000 * 1.02^2 + 1000 * 1.02 + 1000) * 1.02 + 1000$
                                           $ = 25000 * 1.02^3 + 1000 * 1.02^2 + 1000 * 1.02^1 + 1000 * 1.02^0$
    for n years from now, the salary will be $25000 * 1.02^n + \frac {1000 (1 - 1.02^n)}{1 - 1.02}$

    ????

\end{large}

\end{document}


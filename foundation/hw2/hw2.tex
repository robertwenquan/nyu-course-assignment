\documentclass{article}
\usepackage[utf8]{inputenc}

\title{CS6003-INET Foundation of Computer Science Problem Set2}
\author{Robert Wen (robert.wen@nyu.edu), NetID: qw476, N12246277}
\date{September 2015}

\begin{document}

\maketitle

1. Prove that A∩B¯¯¯¯¯¯¯¯¯=A¯¯¯∪B¯¯¯ by giving a proof using logical equivalence.
 
In Problems 2–4 mark each statement TRUE or FALSE. Assume that the statement applies to all sets. 

2. A−(B−C)=(A−B)−C.

3. (A−C)−(B−C)=A−B.

4. A∪B¯¯¯¯¯¯¯¯¯¯¯¯∪A¯¯¯=A¯¯¯. 

5. Find ⋂i=1+∞[1−1i,1]. 

6. Suppose A={x,y} and B={x,{x}}. Mark the statement TRUE or FALSE.

{x}⊆A−B.
7. Suppose A={1,2,3,4,5}. Mark the statement TRUE or FALSE.

{{3}}⊆P(A). 
8. Determine whether the set is finite or infinite. If the set is finite, find its size.

{x|x∈N and 4x2−8=0}. 
9. Determine whether the following set is countable or uncountable. If it is countably infinite, exhibit a one-to-one correspondence between the set of positive integers and the set.

The set of irrational numbers between 2√ and π2. 
10. Show that (0,1) has the same cardinality as (0,2).
11. Suppose f:N→N has the rule f(n)=4n+1. Determine whether f is 1-1.

12. Suppose f:N→N has the rule f(n)=4n+1. Determine whether f is onto N.

13. Suppose f:Z→Z has the rule f(n)=3n2−1. Determine whether f is 1-1.

14. Suppose f:Z→Z has the rule f(n)=3n−1. Determine whether f is onto Z.

15. Suppose f:N→N has the rule f(n)=3n2−1. Determine whether f is 1-1.

16. Suppose f:N→N has the rule f(n)=4n2−1. Determine whether f is onto N.

17. Suppose g:R→R where g(x)=⌊x−12⌋.

Is g 1-1?

Is g onto R?

In questions 18–19 suppose g:A→B and f:B→C where A={a,b,c,d},B={1,2,3},C={2,3,6,8}, and g and f are defined by g={(a,2),(b,1),(c,3),(d,2)} and f={(1,8),(2,3),(3,2)}.

18. Find f∘g. 

19. Find f−1.

20. Suppose g:A→B and f:B→C, wheref∘g is 1-1 and g is 1-1. Must f be 1-1?

21. Suppose g:A→B and f:B→C, wheref∘g is 1-1 and f is 1-1. Must g be 1-1?

22. Verify that an=3n+1 is a solution to the recurrence relation an=4an−1−3an−2

23. You take a job that pays $25,000 annually. 

(a) How much do you earn n years from now if you receive a three percent raise each year? 

(b) How much do you earn n years from now if you receive a five percent raise each year? 

(c) How much do you earn n years from now if each year you receive a raise of $1000 plus two percent of your previous year’s salary.

\end{document}

